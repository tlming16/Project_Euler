\documentclass{article}
\usepackage{ctex}
\usepackage{amsmath}
\usepackage{amssymb}
\newcommand \row[1]{$\displaystyle #1$}
\begin{document}
\noindent 
题目15
\begin{equation*}
\begin{split}
f(x)&=2\cos{x} (\sin{x}\cos{\frac{\pi}{3}}+\cos{x}\sin{\frac{\pi}{3}}) -\sqrt{3} \frac{1-\cos{2x}}{2}+\frac{1}{2}\sin{2x} \\
&=\sin{x}\cos{x}+\sqrt{3}\cos^2{x}-\sqrt{3}(\frac{1}{2}-\frac{1}{2}\cos{2x}) +\frac{1}{2}\sin{2x} \\
&= \sin{2x}+\sqrt{3}\frac{\cos{2x}+1}{2} -\frac{\sqrt{3}}{2}+
\frac{\sqrt{3}}{2}\cos{2x}\\
&= \sin{2x} +\sqrt{3}\cos{2x} \\
&= 2\sin{(2x+\frac{\pi}{3})}
\end{split}
\end{equation*}
所以 函数周期为\row{\pi},振幅为\row{2},初相为\row{\frac{\pi}{3}},
单调递增区间 \row{-\frac{\pi}{2}+2k\pi \le 2x+\frac{\pi}{3}\le \frac{\pi}{2}+2k\pi}\row{\Longrightarrow x\in (-\frac{5\pi}{12}+k\pi,\frac{\pi}{12}+k\pi),k\in Z}

题目17 \\
(1) 最大温差:$30^{\circ}C -10^{\circ}C=20^{\circ}C$\\ 
(2) 
\begin{equation*}
\begin{cases}
\begin{split}
A+b&=30 \\
-A+b&=10 
\end{split}
\end{cases}
\Longrightarrow 
\begin{cases}
\begin{split}
A=10 \\
b=20
\end{split}
\end{cases}
\end{equation*}
周期\row{T=2\times(14-6)=16\Longrightarrow \omega=\frac{2\pi}{16}=\frac{\pi}{8}}
将\row{(16,30)} 代入函数得\row{10\sin{(\frac{\pi}{8}\times 16+\phi)}+20=30}\row{\Longrightarrow \phi=-\frac{5\pi}{4}+2k\pi \Longrightarrow 
\phi=\frac{3\pi}{4}}

题 18\\
周期\row{T=2\times(\frac{11\pi}{12}-\frac{5\pi}{12})=\pi \Longrightarrow \omega=\frac{2\pi}{T}=2}, 
\row{A=3},将\row{(\frac{5\pi}{12},3)}代入\row{y=3\sin{(2x+\phi)}}有
\[2\times \frac{5\pi}{12}+\phi=\frac{\pi}{2}+2k\pi \Longrightarrow \phi=-\frac{\pi}{3}+3k\pi,(k\in Z)\] 
所以函数表达式为\[y=3\sin{(2x-\frac{\pi}{3})}\]

题目19 
(1) 
\begin{equation*}
\begin{split}
f(x) &=\sqrt{3}\cos^2{\omega x}+\frac{1}{2}\sin{2\omega x} +a\\
     &=\sqrt{3}\frac{\cos{2\omega x}+1}{2} +\frac{1}{2}\sin{2\omega x}+a \\
     &= \frac{\sqrt{3}}{2}\cos{2\omega x}+\frac{1}{2}\sin{2\omega x} +\frac{\sqrt{3}}{2}+a \\
     &=\sin{(2\omega x+\frac{\pi}{3})} +\frac{\sqrt{3}}{2}+a
\end{split}
\end{equation*}
\[2\omega \times \frac{\pi}{6}+\frac{\pi}{3}=\frac{\pi}{2} \Longrightarrow \omega=\frac{1}{2}\]
(2) \[f(x)=\sin{(x+\frac{\pi}{3})}+\frac{\sqrt{3}}{2}+a,x\in[-\frac{\pi}{3},\frac{5\pi}{6}] \Longrightarrow x+\frac{\pi}{3}\in[0,\frac{7\pi}{6}]\]
最小值$$\sin{\frac{7\pi}{6}}+\frac{\sqrt{3}}{2}+a=-\frac{1}{2}+\frac{\sqrt{3}}{2}+a=\sqrt{3}$$
所以 $$a=\frac{\sqrt{3}+1}{2}$$

21 题
(1) \[\sin{x}+\cos{x}=\sqrt{2}\sin{(x+\frac{\pi}{4})}\in[-\sqrt{2},\sqrt{2}] \Longrightarrow a\in[-\sqrt{2},\sqrt{2}]\text{\quad 有解}\]
(2) 当$x\in [0,\pi]$时 
\[ x+\frac{\pi}{4} \in [\frac{\pi}{4},\frac{5\pi}{4}] \text{\quad 有两解} \Longrightarrow x_1\in [0,\frac{\pi}{4}), 
x_2\in (\frac{\pi}{4},\frac{\pi}{2}] \Longrightarrow a\in [1,\sqrt{2}) \text{,此时两根关于} x=\frac{\pi}{4} \text{对称}\]
即两根之和为\row{\frac{\pi}{2}}
\end{document}