\documentclass[12pt,answers]{exam} 
\usepackage{ctex}
\usepackage[fleqn]{amsmath}
\usepackage{graphicx}
\graphicspath{{fig/pdf/}}
\usepackage{caption}
\usepackage[T1]{fontenc}
\usepackage{fourier}
\usepackage{amssymb}
\usepackage{gensymb}
\usepackage{latexsym}
\usepackage{tikz}
\newcommand*{\circled}[1]{\lower.7ex\hbox{\tikz\draw (0pt, 0pt)%
    circle (.5em) node {\makebox[1em][c]{\small #1}};}}

\renewcommand{\thepartno}{\arabic{partno}}
\renewcommand{\solutiontitle}{\noindent\textbf{解:}\par\noindent}

\begin{document}
\pagestyle{plain}
\begin{center}
{\heiti{高一第二学期数学周周练一}}
\end{center}

\begin{questions}
\fullwidth{\heiti 一,填空题}
\question 
在半径为2米的圆中,$120^{\circ}$的圆心角所对的弧长为\fillin [$\displaystyle \frac{4\pi}{3}$]米
\question 
已知扇形的圆心角为$2$,半径为$1$,弦长为\fillin[$2\sin{1}$]
\question 
若角$\alpha$ 的终边和函数$y=|x|$的图像重合,则$\alpha$的集合是\\
\fillin[$\displaystyle \left\{ \alpha\mid \alpha =\frac{\pi}{4}+2k\pi,k\in Z \right\} \cup \left\{ \alpha\mid \alpha=\frac{3\pi}{4}+2k\pi,k\in Z \right\}$][10cm]

\question 
若$\displaystyle \sin{\frac{\theta}{2}}=\frac{3}{5},\cos{\frac{\theta}{2}}=-\frac{4}{5}$,则$\theta$的终边在第\fillin[四]象限。

\question 
已知 $|\sin{\alpha}|=|\cos{\alpha}|$,则角$\alpha$终边表示的集合为\fillin[$\displaystyle \left\{ \alpha\mid \alpha=\frac{\pi}{4}+\frac{k\pi}{2},k\in Z\right\}$][5cm]
\question 
已知$\theta$ 是第一象限角,且满足$\displaystyle |\sin{\frac{\theta}{2}}|=-\sin{\frac{\theta}{2}}$,则$\frac{\theta}{2}$的终边在第\fillin[三]象限

\question 
若$\displaystyle \cos{\alpha}=-\frac{\sqrt{3}}{2}$,且$\alpha$的终边过点$P(x,2)$,则$\alpha$是第\fillin[二]象限角,$x=$ \fillin[$-2\sqrt{3}$]

\question 

已知$\tan{\alpha}=3$,$\displaystyle \frac{\sin{\alpha}+2\cos{\alpha}}{3\cos{\alpha}+\sin{\alpha}}$=\fillin[$\displaystyle \frac{5}{6}$]

\question 
化简:$\displaystyle \frac{2\sin^2{\alpha}-1}{1-2\cos^2{\alpha}}=$ \fillin[$1$]

\question 
已知 $\displaystyle \sin{\alpha}=-\frac{\sqrt{3}}{2},\alpha \in [-2\pi,0]$,则$\alpha=$ \fillin[$\displaystyle -\frac{\pi}{3}\text{或} -\frac{5\pi}{3}$]

\question 
已知 $\displaystyle \sin{\alpha}\cos{\alpha}=\frac{1}{8},\frac{\pi}{4}<\alpha<\frac{\pi}{2}$,则$\cos{\alpha}-\sin{\alpha}=$ \fillin[$\displaystyle -\frac{\sqrt{3}}{2}$]

\question 
设$\alpha$ 是第三象限角,且 $2\sec^2{\alpha}+3\sec{\alpha}-2=0$,则$\cot{\alpha}=$ \fillin[$\displaystyle \frac{\sqrt{3}}{3}$]

\fullwidth{\heiti 二,选择题}
\question 
已知$\cos{130^\circ}=a,\tan{50^\circ}$等于\fillin[D]\\
\begin{oneparchoices}
\choice $\displaystyle \frac{\sqrt{1-a^2}}{a}$
\choice $\displaystyle \pm \frac{\sqrt{1-a^2}}{a}$ 
\choice $\displaystyle \pm \frac{a}{\sqrt{1-a^2}}$
\choice $\displaystyle -\frac{\sqrt{1-a^2}}{a}$ 
\end{oneparchoices}

\question 
给出下列命题,正确的命题个数是\fillin[B] \\
\circled{1}存在一个锐角$\alpha$,使得$\sin{\alpha}=\cos{\alpha}<1$ ;
\circled{2}存在一个锐角$\alpha$,使得$\sin{\alpha}+\cos{\alpha}<1$;
\circled{3} 存在一个角$\alpha$,使得$\displaystyle \sin{\alpha}+\cos{\alpha}=\frac{3}{2}$;
\circled{4} 如果$\alpha,\beta$ 都是第一象限的角,且$\alpha>\beta$,则$\tan{\alpha}>\tan{\beta}$\\
\begin{oneparchoices}
\choice $1$  
\choice $2$
\choice $3$
\choice $4$
\end{oneparchoices}

\question 一个半径为$R$的扇形,周长伟$4R$,则这个扇形所含弓形的面积是\fillin[D]  \\
\begin{oneparchoices}
\choice $\displaystyle \frac{1}{2}R^2$  
\choice $\displaystyle \frac{1}{2}R^2\sin{1}\cos{1}$
\choice $\displaystyle \frac{1}{2}(2-\sin{1}\cos{1})R^2$
\choice $R^2-\sin{1}\cos{1}R^2$
\end{oneparchoices}
\question 
若扇形的圆心角为$60^{\circ}$,半径为$\alpha$,则扇形内切圆与扇形面积之比为\fillin[C] \\ 
\begin{oneparchoices}
\choice $1:2$  
\choice $1:3$
\choice $2:3$
\choice $3:4$
\end{oneparchoices}

\fullwidth{\heiti 三,解答题}
\question
已知$\displaystyle \cos{(\pi+\alpha)}=-\frac{1}{3},\frac{3\pi}{3}<\alpha<2\pi$,求$\sin(2\pi-\alpha),\tan(\alpha-3\pi)$ 的值\\
解:\\
 $\because \displaystyle \frac{3\pi}{3}<\alpha<2\pi,\cos{(\pi+\alpha)}=-\frac{1}{3},\Rightarrow \cos{\alpha}=\frac{1}{3},\sin{\alpha}=-\frac{2\sqrt{2}}{3}$\\ 
$\displaystyle \therefore \sin(2\pi-\alpha)=-\sin{\alpha}=\frac{2\sqrt{2}}{3},\tan(\alpha-3\pi)=\tan{\alpha} =\frac{\sin{\alpha}}{\cos{\alpha}}=-2\sqrt{2}$

\question 
已知$\displaystyle \tan{\alpha}=\frac{1}{2}$,求$\sin^2{\alpha}+3\sin{\alpha}\cos{\alpha}-\cos^2{\alpha}$的值。\\
解:
\begin{equation*}
\begin{split}
\sin^2{\alpha}+3\sin{\alpha}\cos{\alpha}-\cos^2{\alpha}&=\frac{\sin^2{\alpha}+3\sin{\alpha}\cos{\alpha}-\cos^2{\alpha}}{\sin^2{\alpha}+\cos^2{\alpha}}\\
																				  &= \frac{\tan^2{\alpha}+3\tan{\alpha}-1}{1+\tan^2{\alpha}}\\
                                                       & =\frac{\frac{1}{4}+\frac{3}{2}-1}{1+\frac{1}{4}}\\
																					&=\frac{3}{5}
\end{split}
\end{equation*}
\question 
已知$\sin{\alpha},\cos{\alpha}$是方程 $8x^2+6kx+2k+1=0$的两个实根,求$k$的值。\\
解:\\
\begin{equation*}
\begin{cases}
\begin{split}
\Delta& =36k^2 -32(2k+1) =4(9k^2-16k-8)\ge 0 \\
%x_1&= \frac{1}{8} \left(-\sqrt{9 k^2-16 k-8}-3 k\right)\\
%x_2&=\frac{1}{8} \left(\sqrt{9 k^2-16 k-8}-3 k\right)\\
x_1& +x_2= -\frac{3k}{4} \\
x_1& x_2 =\frac{2k+1}{8}\\
1&=x_1^2 +x_2^2=(x_1+x_2)^2-2x_1x_2 =\frac{9k^2}{16}-\frac{2k+1}{4}
\end{split}
\end{cases}
\Rightarrow k=-\frac{10}{9}(\text{舍去}k=2)
\end{equation*}
\question 
化简: $\displaystyle \frac{1-\sin^4{\alpha}-\cos^4{\alpha}}{1-\sin^6{\alpha}-\cos^6{\alpha}}$\\ 
解:\\
$\displaystyle \sin^4{\alpha}+\cos^4{\alpha} =(\sin^2{\alpha}+\cos^2{\alpha})^2 -2\cos^2{\alpha}\sin^2{\alpha} = 1-2\cos^2{\alpha}\sin^2{\alpha}$ \\
$\displaystyle \sin^6{\alpha}+\cos^6{\alpha} =(\sin^2{\alpha}+\cos^2{\alpha})(\sin^4{\alpha}+\cos^4{\alpha}-\sin^2{\alpha}\cos^2{\alpha})= 1-3\cos^2{\alpha}\sin^2{\alpha}$\\
$\displaystyle \Rightarrow \frac{1-\sin^4{\alpha}-\cos^4{\alpha}}{1-\sin^6{\alpha}-\cos^6{\alpha}}=\frac{2\cos^2{\alpha}\sin^2{\alpha}}{3\cos^2{\alpha}\sin^2{\alpha}}=\frac{2}{3}$
\question 
已知一个扇形的周长为40,求它的面积$S$的最大值,并求面积最大时的扇形的半径和圆心角。\\
解:设扇形的半径为$r$,圆心角为$\alpha$,圆心角所对弧长为$l$,则有
\begin{equation*}
\begin{cases}
\begin{split}
l=r\alpha\\
l+2r&=40  \\
S=\frac{1}{2}lr&=\frac{1}{2}(40-2r)r= -(r-10)^2+100
\end{split}
\end{cases}
\Rightarrow S_{max}=100,\text{此时} r=10,\alpha = \frac{40-2r}{r}=2
\end{equation*}
\end{questions}
\end{document}