\documentclass[12pt,answers]{exam} 
\usepackage{ctex}
\usepackage[fleqn]{amsmath}
\usepackage{graphicx}
\graphicspath{{fig/pdf/}}
\usepackage{caption}
\usepackage[T1]{fontenc}
\usepackage{fourier}
\usepackage{amssymb}
\usepackage{gensymb}
\usepackage{latexsym}
\usepackage{tikz}
\newcommand*{\circled}[1]{\lower.7ex\hbox{\tikz\draw (0pt, 0pt)%
    circle (.5em) node {\makebox[1em][c]{\small #1}};}}

\renewcommand{\thepartno}{\arabic{partno}}
\renewcommand{\solutiontitle}{\noindent\textbf{解:}\par\noindent}

\begin{document}
\pagestyle{plain}
\begin{center}
{\heiti{高一第二学期数学周周练二}}
\end{center}

\begin{questions}
\fullwidth{\heiti 一,填空题}
\question 
已知$\alpha$是第二象限角,$P(x,\sqrt{5})$ 为其终边上一点,且$\displaystyle \cos{\alpha}=\frac{\sqrt{2}}{4}x$,则$\sin{\alpha}=$\fillin[$\displaystyle \frac{\sqrt{10}}{4}$]
\question 
若$\sin{\alpha}>0,\tan{\alpha}>0$,则$2\alpha$的终边落在\fillin[第一象限,第二象限,$y$轴正半轴][8cm]
\question 
已知$\displaystyle \sin{(\alpha+\frac{\pi}{4})}=\frac{4}{5},\frac{\pi}{4}<\alpha<\frac{3\pi}{4}$,则$\cos{2\alpha}=$\fillin[$\displaystyle -\frac{24}{25}$]

\question 
在$\Delta ABC$中,若$2\cos{B}\sin{A}=\sin{C}$,则$\Delta ABC$的形状一定是\fillin[ 等腰三角形]

\question 
已知$\displaystyle \cos{\theta}=\frac{1}{3},\theta \in(0,\pi)$,则$\cos{(\pi+2\theta)}=$\fillin[$\displaystyle \frac{7}{9}$]

\question 
已知$\displaystyle \sin{\alpha}+\cos{\alpha}=\frac{7}{13},0<\alpha<\pi$,则$\tan{\alpha}=$ 
\fillin[$\displaystyle -\frac{12}{5}$]

\question 
若$\tan{(7\pi+\alpha)}=a$,则$\displaystyle \frac{\sin{(\alpha-3\pi)}+\cos{(\pi-\alpha)}}{\sin{(-\alpha)}-\cos{(\pi+\alpha)}}=$\fillin[$\displaystyle \frac{a+1}{a-1}$]

\question 

已知$\tan{\alpha}=2,\tan{(\beta-\alpha)}=3$,则$\tan{(\beta-2\alpha)}=$\fillin[$\displaystyle \frac{1}{7}$]

\question 
已知$\alpha$为第二象限角,化简$\displaystyle \sqrt{\frac{1+\sin{\alpha}}{1-\sin{\alpha}}}-\sqrt{\frac{1-\sin{\alpha}}{1+\sin{\alpha}}}=$\fillin[$\displaystyle  2\tan{\alpha}$]

\question 
已知 $\displaystyle \sin{2\alpha}=-\frac{24}{25},\alpha \in (-\frac{\pi}{4},0)$,则$\sin\alpha+\cos\alpha=$ \fillin[$\displaystyle \frac{1}{5}$]

\question 
已知$\displaystyle \tan{\alpha}=\sqrt{3},\alpha \in(\frac{\pi}{2},\frac{3\pi}{2})$,则$\cos\alpha-\sin\alpha=$\fillin[$\displaystyle \frac{\sqrt{3}-1}{2}$]

\question 
已知 $\tan{\theta}=2$,则$\displaystyle \frac{1}{4}\sin^2\theta+\frac{1}{2}\sin{2\theta}$的值\fillin[$\displaystyle \frac{3}{5}$]

\fullwidth{\heiti 二,选择题}
\question 
角$\theta$ 为第三象限角的充分必要条件是\fillin[B]\\
\begin{oneparchoices}
\choice $\displaystyle \sin\theta<0 \text{或} \tan\theta>0$
\choice $\displaystyle \sin\theta<0 \text{且} \tan\theta>0$ 
\choice $\displaystyle \sin\theta\tan\theta<0$
\choice $\displaystyle \sin\theta\tan\theta>0$ 
\end{oneparchoices}

\question 
若 $\displaystyle \alpha+\beta=\frac{3\pi}{4}$则$(1-\tan\alpha)(1-\tan\beta)=$\fillin[B]\\
\begin{oneparchoices}
\choice $1$  
\choice $2$
\choice $3$
\choice $4$
\end{oneparchoices}

\question 
$\displaystyle f(\frac{\pi}{2}-x)=\cos{x}$,则$\displaystyle f(\frac{\pi}{3})=$\fillin[B] \\
\begin{oneparchoices}
\choice $\displaystyle \frac{1}{2}$  
\choice $\displaystyle \frac{\sqrt{3}}{2}$
\choice $\displaystyle -\frac{\sqrt{3}}{2}$
\choice $\displaystyle -\frac{1}{2}$
\end{oneparchoices}
\question 
当$\displaystyle -\frac{\pi}{2}\le x\le \frac{\pi}{2}$,函数$f(x)=\sin{x}+\sqrt{3}\cos{x}$的\fillin[D]
\begin{choices}
\choice 最大值是1,最小值是-1  
\choice 最大值是1,最小值是$\displaystyle \frac{1}{2}$
\choice 最大值是2,最小值是-2
\choice 最大值是2,最小值是-1
\end{choices}

\fullwidth{\heiti 三,解答题}
\question
设$\displaystyle \theta\in(0,\pi),\sin\theta+\cos\theta=\frac{1}{2},$
\begin{parts}
\part 求$\sin^4\theta+\cos^4\theta$
\part 求$\cos{2\theta}$的值
\end{parts}
解:
\begin{parts}
\part 
$\sin^4\theta+\cos^4\theta=(\sin^2\theta+\cos^2\theta)^2 - 2\sin^2\theta\cos^2\theta=1-2\sin^2\theta\cos^2\theta$ \\
$   \displaystyle 2\sin\theta\cos\theta = (\sin\theta+\cos\theta)^2 -\sin^2\theta-\cos^2\theta \Rightarrow \sin\theta\cos\theta =-\frac{3}{8}$\\
因此 $\displaystyle \sin^4\theta+\cos^4\theta = 1-2\sin^2\theta\cos^2\theta = \frac{23}{32}$
\part 
$\displaystyle \cos{2\theta} = \left(\cos\theta+\sin\theta\right)\left(\cos\theta-\sin\theta\right)=\frac{1}{2}\left(\cos\theta -\sin\theta\right)$ \\
而$\displaystyle (\cos\theta -\sin\theta)^2 = \sin^2\theta+\cos^2\theta -2\sin\theta\cos\theta =\frac{7}{4}\Rightarrow (\cos\theta-\sin\theta)=\pm \frac{\sqrt{7}}{2}$\\
$\displaystyle \theta\in(0,\pi),\sin\theta>0 \Rightarrow (\cos\theta-\sin\theta)=- \frac{\sqrt{7}}{2} $ \\
所以 $\cos{2\theta}=-\frac{\sqrt{7}}{4}$
\end{parts}
\question 
证明恒等式:$\displaystyle \tan\alpha-\cot\alpha=\frac{1-2\cos^2\alpha}{\sin\alpha\cos\alpha}$\\
证: 
\begin{equation*}
\begin{split}
\tan\alpha -\cot\alpha &= \frac{\sin\alpha}{\cos\alpha} -\frac{\cos\alpha}{\sin\alpha} \\
                       &= \frac{\sin^2\alpha-\cos^2\alpha}{\cos\alpha\sin\alpha}\\
                       & = \frac{1-2\cos^2\alpha}{\cos\alpha\sin\alpha}
\end{split}
\end{equation*}

\question
已知$\alpha,\beta \in (0,\pi)$,且$\displaystyle \tan(\alpha-\beta)=\frac{1}{2},\tan\beta=-\frac{1}{7}$,求$2\alpha-\beta$的值\\
解:$\displaystyle \tan{(2\alpha-\beta)} =\tan(2\alpha-2\beta +\beta)=\frac{\tan(2\alpha-2\beta)+\tan\beta}{1-\tan(2\alpha-2\beta)\tan\beta}$  \\
$\displaystyle \tan{(2\alpha-2\beta)} =\frac{2\tan(\alpha-\beta)}{1-\tan^2(\alpha-\beta)}=\frac{4}{3}\Rightarrow \tan(2\alpha-\beta)= \frac{\frac{4}{3}-\frac{1}{7}}{1+\frac{4}{3}\times\frac{1}{7}}=1$\\
$\displaystyle \Rightarrow 2\alpha-\beta =\frac{\pi}{4}+k\pi$,$k$待定 \\
由$\displaystyle \tan\alpha =\frac{\tan(\alpha-\beta)+\tan\beta}{1-\tan(\alpha-\beta)\tan\beta}=\frac{1}{3}\Rightarrow \alpha \in (0,\frac{\pi}{4})$\\
由$\displaystyle \tan\beta<0\Rightarrow \beta \in (\frac{\pi}{2},\pi)$\\
$\displaystyle \tan(\alpha-\beta)>0 \Rightarrow \alpha-\beta\in (-\pi,-\frac{3\pi}{4})$ \\
因此$\displaystyle 2\alpha-\beta =\alpha +\alpha -\beta =-\frac{3\pi}{4}$
\question 
已知 $\displaystyle f(x)=\frac{\sin{2x}-\cos{2x}+1}{1+\cot{x}}$
\begin{parts}
\part 化简 $f(x)$
\part 若$\displaystyle \sin(x+\frac{\pi}{4})=\frac{3}{5}$,且$\displaystyle \frac{\pi}{4}<x<\frac{3\pi}{4}$,求$f(x)$的值。
\end{parts}
解:
\begin{parts}
\part  
\begin{equation*}
\begin{split}
f(x) =&\frac{\sin{x}((\sin{x}+\cos{x})^2 -(\cos{x}+\sin{x})(\cos{x}-\sin{x}))}{\sin{x}+\cos{x}}\\
     =& \sin{x} (\sin{x}+\cos{x} -\cos{x}+\sin{x}) \\
     =& 2\sin^2{x}
\end{split}
\end{equation*}
\part $\displaystyle \sin(2x+\frac{\pi}{2}) =2\sin(x+\frac{\pi}{4})\cos(x+\frac{\pi}{4})$ .由$\displaystyle \frac{\pi}{4}<x<\frac{3\pi}{4}\Rightarrow \cos(x+\frac{\pi}{4})=-\frac{4}{5}$\\
因此$\displaystyle f(x)=2\sin^2{x}=1-\cos(2x)=1-\sin(2x+\frac{\pi}{2}) =1 +\frac{24}{25}=\frac{49}{25}$ 
\end{parts}
\end{questions}
\end{document}