\documentclass{article}
\usepackage{amsmath}
\usepackage{ctex} 
\usepackage[table]{xcolor}
\usepackage{color}
\usepackage{colortbl}
\begin{document}

\begin{table}
\caption{反正切}
\begin{center}
\begin{tabular}{|l|>{\columncolor[gray]{1}[1pt]} c|}
\hline
定义域 & \textcolor{blue}{${R}$} \\
\hline 
值域 & \textcolor{blue}{$\displaystyle \left(-\frac{\pi}{2},\frac{\pi}{2}\right)$ }\\ \hline 
单调性 & \textcolor{blue}{在$R$ 上单调递增 }\\ \hline 
奇偶性 & \textcolor{blue}{奇函数 }\\ \hline 
分段 & \textcolor{blue}{$\begin{cases}\displaystyle x>0,y\in\left(0,\frac{\pi}{2}\right)\\
\displaystyle x=0,y=0\\  \displaystyle x<0,y\in \left(-\frac{\pi}{2},0\right) \end{cases}$ }\\ \hline
\end{tabular}
\end{center}
\end{table}
\begin{table}
\caption{反余弦}
\begin{center}
\begin{tabular}{|l|>{\columncolor[gray]{1}[1pt]} c|}
\hline
定义域 & \textcolor{blue}{$[-1,1]$} \\
\hline 
值域 & \textcolor{blue}{$\displaystyle [0,\pi]$ }\\ \hline 
单调性 & \textcolor{blue}{在$[-1,1]$ 上单调递减 }\\ \hline 
奇偶性 & \textcolor{blue}{非奇非偶函数 }\\ \hline 
分段 & \textcolor{blue}{$\begin{cases}\displaystyle x\in(0,1),y\in (0,\frac{\pi}{2})\\
	\displaystyle x=0,y=\frac{\pi}{2}\\ 
  \displaystyle x\in(-1,0),y\in( \frac{\pi}{2},\pi ) \end{cases}$ }\\ \hline
\end{tabular}
\end{center}
\end{table}
\end{document}